\documentclass[twocolumn]{article}

% Set for specific document
\def\DOCTITLE{CSC3121 Distributed Systems Coursework 1}
\def\DOCAUTHOR{Dan Nixon (120263697)}
\def\DOCDATE{13/11/2015}

% Set document attributes
\title{\DOCTITLE}
\author{\DOCAUTHOR}
\date{\DOCDATE}

\usepackage{fullpage}
\usepackage{scrextend}
\usepackage{titlesec}
\usepackage{fancyhdr}
\usepackage{amssymb}

% Handle graphics correctly
\ifx\pdftexversion\undefined
\usepackage{graphicx}
% \usepackage[dvips]{graphicx}
\else
\usepackage[pdftex]{graphicx}
\DeclareGraphicsRule{*}{mps}{*}{}
\fi

% Setup headers and footers
\pagestyle{fancy}
\lhead{}
\chead{\DOCTITLE}
\rhead{}
\rfoot{\DOCDATE}
\cfoot{\thepage}
\lfoot{\DOCAUTHOR}

% New page for each section
\newcommand{\sectionbreak}{\clearpage}

% Set header and footer sizes
\renewcommand{\headrulewidth}{0.4pt}
\renewcommand{\footrulewidth}{0.4pt}
\setlength{\headheight}{15.2pt}
\setlength{\headsep}{15.2pt}

\begin{document}

\section{Question 1}

The correctness of the system as defined by the specification can be summarised
as follows: a fire engine should be dispatched if and only if there is at least
one \textit{fire!} message received by computer C which is not followed by a
\textit{fire out} message $2d + e$ time units later. This being the case depends
on the following:

\begin{enumerate}
  \item[1] For a \textit{fire!} message $M_{f}$ and \textit{fire out} message
           $M_{o}$, the relation $M_{f} \Rightarrow M_{o}$ must hold.
  \item[2] In the event of multiple fires, \textit{fire out} messages should
           only negate the corresponding \textit{fire!} message and not those
           caused by other fires.
\end{enumerate}

Both of these conditions can be violated in the current system.

\subsection{Violation of condition 1}
\label{sec:violation_of_condition_1}

This describes the case where the \textit{fire!} message takes the maximum
transfer time ($e$) between computer A and C and all other messages
(\textit{fire!} from A to B, \textit{fire out} from B to C) and events (the fire
being extinguished) happen in a significantly shorter time.

This case is demonstrated in figure \ref{fig:failure_case_1}.

This results in the \textit{fire out} message for a given fire arriving before
the corresponding \textit{fire!} message.

\begin{figure}[h!]
  \centering
  \includegraphics[width=0.4\textwidth]{out/failure_case_1.eps}
  \caption{Violation of condition 1}
  \label{fig:failure_case_1}
\end{figure}

In this case the system fails as a fire engine will be dispatched when the fire
has already been extinguished by the sprinkler system.

\subsection{Violation of condition 2}
\label{sec:violation_of_condition_2}

This describes the case when between computer B sending a \textit{fire out}
message ($E1$) and the same message being received by computer C, a new
\textit{fire!} message ($F2$) is sent by computer A which is received at
computer C before $E1$.

This case is demonstrated in figure \ref{fig:failure_case_2}.

In this case the last message received by the fire service computer C will be
\textit{fire out} and will have arrived within $2d + e$ time units of the last
\textit{fire!} message ($F2$).

\begin{figure}[h!]
  \centering
  \includegraphics[width=0.4\textwidth]{out/failure_case_2.eps}
  \caption{Violation of condition 2}
  \label{fig:failure_case_2}
\end{figure}

In this case the system has failed as $F2$ signals a fire that cannot be
extinguished by the sprinkler system yet the fire service will not dispatch a
fire engine.

\section{Question 2}
\label{sec:question_2}

The use of logical clocks in the system allows the order of messages to be
tracked within the system, this can be used by computer C to determine the
validity of a message based on its timestamp compared to the local logical
clock of computer C.

A brief overview of the operation of such a system is shown in figure
\ref{fig:logical_clocks}.

In this design computer A sends \textit{fire!} messages to computers B and C
when it receives a signal from a fire detection sensor (it does not receive
messages from either computer).

Computer B receives all messages from computer A and sends \textit{fire out}
messages to computer C when it detects that all fires are extinguished.

The local logical clock ($C_{b}$) of computer B is set to $max(C_{b}, T_{m} +
1)$ when message $m$ is received (where $T_{m}$ is the timestamp of message
$m$).

When C receives either a \textit{fire!} or \textit{fire out} message ($m$) it
first examines the timestamp of the message ($T_{m}$), if the timestamp is equal
to or greater than the local logical clock of computer C ($C_{c}$) then the
message is processed as normal and $C_{c}$ is set such that $C_{c} = max(C_{c},
T_{m} + 1)$, otherwise if $T_{m} < C_{c}$ the message is discarded as computer C
will have already received more recent information than that described by $m$.

\begin{figure}[h!]
  \centering
  \includegraphics[width=0.4\textwidth]{out/logical_clocks.eps}
  \caption{Use of logical clocks}
  \label{fig:logical_clocks}
\end{figure}

In this proposed system the clock starting value of computer A ($C_{a}$) is set
to 10 and the increment ($I_{a}$) set to 10. The starting values of computers B
and C ($C_{b}$ and $C_{c}$) are both set to 0 and the increments ($I_{b}$ and
$I_{c}$) are set to 1.

This is to ensure that timestamps for subsequent fires are significantly larger
than the current local logical clocks on both computers B and C for the purpose
for the purpose of demonstration.

\subsection{Correctness of proposed design}

The correctness of this system can be demonstrated by examining the possible
cases where a message can arrive at computer C out of the intended order.

These cases are listed below, the notation \textit{fire!} $\rhd$ \textit{fire
out} denotes an out of order \textit{fire out} message being received when the
last valid message received was \textit{fire!}.

\begin{description}
  \item[\textit{fire!} $\rhd$ \textit{fire!}] \hfill \\
    No change to state of fire service as they have already been alerted to the
    presence of a fire in the last \textit{fire!} message.

  \item[\textit{fire!} $\rhd$ \textit{fire out}] \hfill \\
    As the \textit{fire out} message will have a lower timestamp to the last
    received \textit{fire!} message it can be assumed hat the two messages are
    associated with two separate fires (one of which has been extinguished and
    the other is still ongoing).

    Therefore the \textit{fire out} message can be discarded as the fire service
    will still need to respond to the more recent fire, alerted to by the last
    received \textit{fire!} message.

  \item[\textit{fire out} $\rhd$ \textit{fire out}] \hfill \\
    No change to state of fire service as they have already been alerted to the
    fire being extinguished in the last \textit{fire out} message.
    The second, out of order \textit{fire out} message with a lower timestamp
    is therefore safe to discard.

  \item[\textit{fire out} $\rhd$ \textit{fire!}] \hfill \\
    As the \textit{fire out} message has a higher timestamp than the
    \textit{fire!} message it can be assumed that the \textit{fire!} message
    could be from any number of fires that started before the \textit{fire out}
    message was sent.
    In this case the \textit{fire!} message can be safely ignored as the fire it
    is alerting to has already been extinguished, alerted to by the last
    received \textit{fire out!} message.
\end{description}

\subsection{The value of $I_{a}$}

The value by which the local logical clock of computer A increments is important
to the correctness of this proposed design, figure \ref{fig:logical_clocks_ia_1}
demonstrates how with $I_{a} = 1$ this system fails by discarding valid
\textit{fire!} messages.

\begin{figure}[h!]
  \centering
  \includegraphics[width=0.4\textwidth]{out/logical_clocks_ia_1.eps}
  \caption{Proposed solution failure with $I_{a} = 1$}
  \label{fig:logical_clocks_ia_1}
\end{figure}

This occurs as the clock of computer A does not allow for the clock of computer
C being incremented by \textit{fire out} messages from computer C.

This is resolved by using a larger $I_{a}$ on computer A than on the other two
computers, as demonstrated in figure \ref{fig:logical_clocks_ia_2}.

\begin{figure}[h!]
  \centering
  \includegraphics[width=0.4\textwidth]{out/logical_clocks_ia_2.eps}
  \caption{Proposed solution with $I_{a} = 10$}
  \label{fig:logical_clocks_ia_2}
\end{figure}

This defines the following condition on $I_{a}$:

\begin{equation}
  I_{a} > 2
  \label{eq:ia_condition}
\end{equation}

This is derived form that fact that two events will happen on computer B between
the \textit{fire!} message being sent by computer A and the \textit{fire out}
message being received by computer C (in the event that the fire is
extinguished).

In all other demonstrations in the design $I_{a} = 10$.

\subsection{Condition 1}

Figure \ref{fig:logical_clocks_fc1} demonstrates how the new system handles the
failure case described in section \ref{sec:violation_of_condition_1}.

In this case the \textit{fire!} message $F1$ does not reach computer C until
after the \textit{fire out} from computer B has already been received.

As a \textit{fire out} message will always have a higher timestamp than the
\textit{fire!} message indicating the detection of the same fire, the
\textit{fire!} message can safely be discarded assuming its timestamp is less
than the local logical clock of computer C ($C_{c}$).

\begin{figure}[h!]
  \centering
  \includegraphics[width=0.4\textwidth]{out/logical_clocks_fc1.eps}
  \caption{Possible violation of condition 1}
  \label{fig:logical_clocks_fc1}
\end{figure}

\subsection{Condition 2}

Figure \ref{fig:logical_clocks_fc2} demonstrates how the new system handles the
failure case described in section \ref{sec:violation_of_condition_2}.

In this case the \textit{fire out} message $E1$ does not reach computer C until
after another \textit{fire!} message $F2$ has already been received by computer
C indicating the detection of a new fire that cannot be extinguished by the
sprinklers.

Here the \textit{fire out} message $E1$ is discarded doe to the \textit{fire!}
message $F2$ being received by computer C first, thus setting the local logical
clock of computer C to a higher timestamp then message $E1$.

\begin{figure}[h!]
  \centering
  \includegraphics[width=0.4\textwidth]{out/logical_clocks_fc2.eps}
  \caption{Possible violation of condition 2}
  \label{fig:logical_clocks_fc2}
\end{figure}

Discarding a \textit{fire out} message in this was can be considered safe as
providing that the last message to be received by computer C is a \textit{fire!}
message originating from the detection of a new fire that started after the
first was extinguished then a fire engine will still be required (unless the
delivery of another \textit{fire out} message follows).

\section{Question 3}

The design described in section \ref{sec:question_2} can be used with physical
clocks without modification on the assumption that the clocks are synchronised
within an error, $\epsilon$, that satisfies the condition:

\begin{equation}
  \epsilon < min((1-k) \mu_{m}, \mu_{f})
  \label{eq:epsilon}
\end{equation}

where $\mu_{m}$ is the minimum time a message can take to be transferred between
two computers and $\mu_{f}$ is the minimum time between the fire sprinklers
being activated and the detection of all fires being extinguished.

This assumption allows the strong clock condition to hold in the system that
includes both electronic messages between computers and the fire element of the
system:

\begin{equation}
  \forall a, b \in S : \mathrm{if} \: a \rightarrow b : C(a) < C(b)
  \label{eq:strong_clock_condition}
\end{equation}

This condition ensures that all messages arrive with the same happens before
relation as was provided in the design using logical clocks.

\begin{figure}[h!]
  \centering
  \includegraphics[width=0.4\textwidth]{out/physical_clocks.eps}
  \caption{Use of physical clocks ($\epsilon = 0$)}
  \label{fig:physical_clocks}
\end{figure}

All examples use $\epsilon = 0$, while this is not feasible in a real world
system it can be used in this example as equation \ref{eq:epsilon} guarantees
that any two clocks cannot differ by more than that shortest time taken for a
message transfer (both inside and outside of the electronic system, i.e.
including the external fire element).

\subsection{Condition 1}

Figure \ref{fig:physical_clocks_fc1} demonstrates the use of physical clocks
under a failure condition for condition 1 described in section
\ref{sec:violation_of_condition_1}.

In this case the late \textit{fire!} message received by computer C still has a
timestamp lower than the local real time clock on computer C so is discarded.

\begin{figure}[h!]
  \centering
  \includegraphics[width=0.4\textwidth]{out/physical_clocks_fc1.eps}
  \caption{Use of physical clocks under failure condition 1 ($\epsilon = 0$)}
  \label{fig:physical_clocks_fc1}
\end{figure}

\subsection{Condition 2}

Figure \ref{fig:physical_clocks_fc2} demonstrates the use of physical clocks
under a failure condition for condition 2 described in section
\ref{sec:violation_of_condition_2}.

Here the \textit{fire out} message is still being received with a timestamp less
than the local real time clock of computer C, so will be discarded, maintaining
correct operation.

\begin{figure}[h!]
  \centering
  \includegraphics[width=0.4\textwidth]{out/physical_clocks_fc2.eps}
  \caption{Use of physical clocks under failure condition 2 ($\epsilon = 0$)}
  \label{fig:physical_clocks_fc2}
\end{figure}

\end{document}
