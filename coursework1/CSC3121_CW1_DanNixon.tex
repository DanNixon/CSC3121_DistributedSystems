\documentclass[twocolumn]{article}

% Set for specific document
\def\DOCTITLE{CSC3121 Distributed Systems Coursework 1}
\def\DOCAUTHOR{Dan Nixon (120263697)}
\def\DOCDATE{13/11/2015}

% Set document attributes
\title{\DOCTITLE}
\author{\DOCAUTHOR}
\date{\DOCDATE}

\usepackage{fullpage}
\usepackage{scrextend}
\usepackage{titlesec}
\usepackage{fancyhdr}

% Handle graphics correctly
\ifx\pdftexversion\undefined
\usepackage{graphicx}
% \usepackage[dvips]{graphicx}
\else
\usepackage[pdftex]{graphicx}
\DeclareGraphicsRule{*}{mps}{*}{}
\fi

% Setup headers and footers
\pagestyle{fancy}
\lhead{}
\chead{\DOCTITLE}
\rhead{}
\rfoot{\DOCDATE}
\cfoot{\thepage}
\lfoot{\DOCAUTHOR}

% New page for each section
\newcommand{\sectionbreak}{\clearpage}

% Set header and footer sizes
\renewcommand{\headrulewidth}{0.4pt}
\renewcommand{\footrulewidth}{0.4pt}
\setlength{\headheight}{15.2pt}
\setlength{\headsep}{15.2pt}

\begin{document}

\section{Question 1}

The correctness of the system as defined by the specification can be summarised
as follows: a fire engine should be dispatched if and only if there is at least
one \textit{fire!} message received by computer C which is not followed by a
\texttt{fire out} message $2d + e$ time units later. This being the case depends
on the following:

\begin{enumerate}
  \item[1] For a \textit{fire!} message $M_{f}$ and \textit{fire out} message
           $M_{o}$, the relation $M_{f} \Rightarrow M_{o}$ must hold.
  \item[2] In the event of multiple fires, \textit{fire out} messages should
           only negate the corresponding \textit{fire!} message and not those
           caused by other fires.
\end{enumerate}

Both of these conditions can be violated in the current system.

\subsection{Violation of condition 1}

This describes the case where the \textit{fire!} message takes the maximum
transfer time ($e$) between computer A and C and all other messages
(\textit{fire!} from A to B, \textit{fire out} from B to C) and events (the fire
being extinguished) happen in a significantly shorter time.

This results in the \textit{fire out} message for a given fire arriving before
the corresponding \textit{fire!} message.

\begin{figure}[h!]
  \centering
  \includegraphics[width=0.4\textwidth]{out/failure_case_1.eps}
  \caption{Violation of condition 1}
  \label{fig:failure_case_1}
\end{figure}

In this case the system fails as a fire engine will be dispatched when the fire
has already been extinguished by the sprinkler system.

\subsection{Violation of condition 2}

This describes the case when between computer B sending a \textit{fire out}
message ($E1$) and the same message being received by computer C, a new
\textit{fire!} message ($F2$) is sent by computer A which is received at
computer C before $E1$.

In this case the last message received by the fire service computer C will be
\textit{fire out} and will have arrived within $2d + e$ time units of the last
\textit{fire!} message ($F2$).

\begin{figure}[h!]
  \centering
  \includegraphics[width=0.4\textwidth]{out/failure_case_2.eps}
  \caption{Violation of condition 2}
  \label{fig:failure_case_2}
\end{figure}

In this case the system has failed as $F2$ signals a fire that cannot be
extinguished by the sprinkler system yet the fire service will not dispatch a
fire engine.

\section{Question 2}

TODO

\section{Question 3}

TODO

\end{document}
